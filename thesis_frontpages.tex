\NTUtitlepage  % 產生論文封面

\newpage
\setcounter{page}{1}
\pagenumbering{roman}
%
\NTUoralpage  % 產生口試委員會審定書

\mydoublespacing
\begin{acknowledgement} %誌謝
%TODO
兩年過去了。

在準備畢業之際,原以為自己能歡欣地將心中的感激好好地表達出來。回首這兩年的故事,其實我也走了一段路。

感謝李琳山老師。有了您的拔擢,讓我從一個徬徨的大學青年,走到現在能夠一窺研究瀚海。您所傳達的研究理念、教學模式還有將知識組織的功夫,是學生望塵莫及,卻又深深感動著的學問。在無數的懇談下,無論是在您的辦公室、明達館的餐廳、布里斯本的夜晚以及出遊時的遊覽車上,總能感受到您誠摯的關懷,讓我在碩士兩年期間,可以稍稍獲得了長輩的肯定,更能勇往直前。

感謝李宏毅老師。您擁有著新鮮老師的熱血和最新科技發展的知識能力,談吐充滿大智慧與哲學思維。跟在您開得機器學習課程上擔任助教,反而學習得比傳授得多上很多很多。

感謝劉元銘學長,您做事的方法與規劃生活的風格深刻地影響著我,讓我學會自處與獨立。感謝周伯威學長,您擇善固執的個性不斷提醒我們思辨正確的研究方式。感謝葉青峰學長,您的帶領讓我能夠正式進入語音辨識的領域。

感謝廖宜修,你替大家扛了很多重要的工作跟任務,讓我們輕鬆不少;感謝吳彥諶,你總是掌握很多論文的精髓跟資訊,提供很好的意見;感謝楊植翔,你把語料庫整理得非常完整清楚,讓我的研究順利許多;感謝沈昇勳,無數的夜晚你陪伴著大家度過期末作業與專題,成為了很紮實與衷心的陪伴。

感謝實驗室的學弟妹們,在我們煩悶緊張的這一年內,你們毫無怨言地接下我們的工作,規劃出遊、實驗室各項雜務,因為畢業焦頭爛額的我們能夠因此專注在畢業的事情上。之後這個棒子交接給你們,希望531是你們會感到寬心與歸屬的存在,傳承下去。

感謝爸爸,無論我學的是什麼,你都希望我努力做好所有事情,告訴我這才是作為肩膀的本質;感謝媽媽,無論我在做什麼,永遠都希望我能夠輕鬆、放鬆地去面對,希望我休息、希望我睡得安穩。有你的在意讓我更加會保護自己;謝謝大哥,我能走到現在幾乎全部靠著你,你讓我選擇自己的路並且支持下去,讓我知道我的未來可以是自己掌握的。感謝願意讓我繼續在台北奮鬥的家人們。爸爸、媽媽、哥哥,我知道你們永遠支持著我,無論是經濟上、精神上、實體上的陪伴,你們是我最完整的廢墟。

感謝室友們,每天處理完一天的煩悶後,能夠回到住處跟你們打屁聊天打電動爬積分。雖然你們一直吵架又忘記帶鑰匙又帶女朋友回來親熱!

感謝初戀,即便我們已經分開,這份論文或多或少都是在你的陪伴下完成的。

最後,我仍然感謝四年前,那堂禮拜三早上的數位語音處理概論課。我永遠忘記不了,第一次踏入課堂裡,看到李琳山老師站在台上授課的那段時光。打下這段致謝的此刻,我彷彿走入了時光機,回到期中考念著維特比演算法的夜晚、回到專題玩小精靈的時期、回到跟在學長後面弄微軟計畫的日子、回到領取獎學金的剎那、回到澳洲研討會的旅行、回到畢業典禮上老師替我撥穗的那一刻、回到口試的那一天。

因為這段相遇,我走到現在。謝謝老師您。謝謝這段相遇。

兩年過去了,我會走得更好的。
\end{acknowledgement}

\begin{zhAbstract}  %中文摘要
隨著巨量資料的發展,語音辨識相關的處理技術越來越成熟,人們渴望著聲音世代能帶來的魅力。此時,這些技術是可流動的,不再只是先進國家獨有的資源,而是世界上不同地區、各種語言使用者都可以享用的科技。這些不同語言的人類語音,雖各自成體系,卻都擁有一個共同點--都是人類能夠藉以互相理解的訊號媒介,承載著感情、觀念、資訊以及聲音的意義。

本論文探討的,是如何讓世界上不同語言的語料互相輔助學習,使得傳統的單語言語音辨識系統擴增成多語音辨識系統,找出其中潛藏的跨語言知識,希望藉以強化各個語言的語音辨識系統。本論文使用GlobalPhone全球音素語料庫,從純語言知識開始,加入資料導向的方法,最後合併了深層類神經網路中間層,由粗糙到細緻,一步一步探討如何可以將聲學模型中跨語言的共通知識合併起來。

一旦有了多語言辨識系統,深層學習的模型將會變得更為龐大,訓練過程也會更為複雜。為了容納龐大資訊並方便即時使用,本論文亦探討了知識蒸餾的方法,將原本多語音辨識系統的龐大模型,濃縮在較小的模型裡,成功提煉出更豐富的跨語言概括化資訊,幫助多語言語音辨識系統變得更加準確。

\end{zhAbstract}

{
%\zhKaiFont
\mysinglespacing\selectfont
\tableofcontents %目錄

\listoffigures  %圖目錄

\listoftables  %表目錄
\par
}

\newpage
\setcounter{page}{1}
\pagenumbering{arabic}
